\chapter{Job}

%--------1---------2---------3---------4---------5---------6---------7---------8
L'\,esempio mostrato nell'\,introduzione è eccessivamente semplice, in generale
lo step viene lanciato con una classe aggiuntiva che fornisce i parametri usati
dallo step per pilotare l'\,esecuzione del codice.
%--------1---------2---------3---------4---------5---------6---------7---------8
Al posto del nome dello step, può essere passata una classe \textsl{statistica}
che fornisce il nome dello step, mette a disposizione dello step dei contatori,
e produce al completamento dello step un report utente di dettaglio.
%--------1---------2---------3---------4---------5---------6---------7---------8
Inoltre il metodo che esegue lo step può restituire un \textsl{returnCode} o
non restituire nulla, in questo caso è sottinteso che viene restituito il codice
di successo se non ci sono eccezioni, e un codice di errore in caso di
eccezioni.
%--------1---------2---------3---------4---------5---------6---------7---------8

%--------1---------2---------3---------4---------5---------6---------7---------8
Considerando tutte le combinazioni sono possibili 8 metodi per lanciare uno
step, vedi lis.~\ref{lst:execPgm}, dove \texttt{P} è la generica classe di
parametri, e \texttt{C} è la classe statistica.
%--------1---------2---------3---------4---------5---------6---------7---------8
\begin{elisting}[!htb]
\begin{javacode}
public interface ExecPgm<S> {
    <P, C extends StatsCount> S execPgm(P p, C c, BiFunction<P, C, Integer> pgm);
    <P> S execPgm(P p, String stepName, ToIntFunction<P> pgm);
    <C extends StatsCount> S execPgm(C c, ToIntFunction<C> pgm);
    S execPgm(String stepName, IntSupplier pgm);
    <P, C extends StatsCount> S execPgm(P p, C c, BiConsumer<P, C> pgm);
    <P> S execPgm(P p, String stepName, Consumer<P> pgm);
    <C extends StatsCount> S execPgm(C c, Consumer<C> pgm);
    S execPgm(String stepName, Runnable pgm);
}
\end{javacode}
\caption{Interfaccia con i metodi di esecuzione di un programma (step)}
\label{lst:execPgm}
\end{elisting}
%--------1---------2---------3---------4---------5---------6---------7---------8

%--------1---------2---------3---------4---------5---------6---------7---------8
In generale un \textit{Job} è composto da più \textit{Step}, e gli step
successivi al primo sono eseguiti in modo condizionale.
Se il primo step è terminato in errore, potrebbe non avere alcun senso
l'\,esecuzione del secondo step.
%--------1---------2---------3---------4---------5---------6---------7---------8
Seguendo l'\,impostazione dei batch mainframe/COBOL, un programma che termina
con successo restituisce il valore $0$, un programma che termina con una
segnalazione (\textit{warning}) restituisce il valore $4$, un programma che
termina in errore con valore maggiore di $4$.
%--------1---------2---------3---------4---------5---------6---------7---------8


%--------1---------2---------3---------4---------5---------6---------7---------8
\begin{elisting}[!htb]
\begin{javacode}
public Integer call() {
    return JCL.getInstance().job("job01")
        .execPgm("step01", step01::run)
        .cond(0,NE).execPgm("step02", step02::run)
        .cond(0,NE).execPgm("step03", step03::run)
        .complete();
}
\end{javacode}
\caption{Esempio di job con step condizionali}
\label{lst:demoStepCond}
\end{elisting}
%--------1---------2---------3---------4---------5---------6---------7---------8
Un esempio di job con esecuzione condizionata è mostrato nel
lis.~\ref{lst:demoStepCond}, l'\,operatore \texttt{cond} permette di indicare
le condizione per le quali il programma successivo non deve essere eseguito in
funzione del codice restituito (\textit{returnCode}) dell'\,ultimo programma
eseguito.
%--------1---------2---------3---------4---------5---------6---------7---------8
Lo step \texttt{step01} viene eseguito in modo incondizionato, se lo step
\texttt{step01} termina con un \textit{returnCode} diverso da zero allora lo
step \texttt{step02} non viene eseguito; analogamente se lo step \texttt{step02}
termina con un \textit{returnCode} diverso da zero, lo step \texttt{step03} non
viene eseguito.
Se lo step \texttt{step01} termina in errore e lo step \texttt{step02} non viene
eseguito, la condizione di esecuzione per lo step \texttt{step03} viene
applicata sul \textit{returnCoce} dell'\,ultimo programma eseguito, lo step
\texttt{step01}, che è terminato in errore, di conseguenza lo step
\texttt{step03} non viene eseguito.
%--------1---------2---------3---------4---------5---------6---------7---------8

\begin{elisting}[!htb]
\begin{javacode}
    Proc<Prm1> proc1 = Proc.create((p, s) -> s
        .execPgm(p, "Step11", step11::run)
        .cond(0,NE).execPgm(p, "Step12", step12::run)
    );
\end{javacode}
\caption{Esempio di definizione di una procedura con una classe parametro}
\label{lst:demoDefProc}
\end{elisting}
%--------1---------2---------3---------4---------5---------6---------7---------8
È possibile raggruppare l'\,esecuzione di una sequenza di programmi definendo
una procedura, come mostrato nel lis.~\ref{lst:demoDefProc}, la procedura così
definita può essere eseguita in modo simile a un programma nel \textit{job}
principale, vedi lis.~\ref{lst:demoUseProc}, o all'\,interno di un'\,altra
procedura.
%--------1---------2---------3---------4---------5---------6---------7---------8
\begin{elisting}[!htb]
\begin{javacode}
public Integer call() {
    Proc<Prm1> proc1 = ...
    return JCL.getInstance().job("job01")
        .execPgm("step01", step01::run)
        .cond(0,NE).execPgm("step02", step02::run)
        .cond(0,EQ,"step01").execProc(prm1, "proc01", proc1)
        .complete();
}
\end{javacode}
\caption{Esempio uso di una procedura nel job}
\label{lst:demoUseProc}
\end{elisting}
%--------1---------2---------3---------4---------5---------6---------7---------8

%--------1---------2---------3---------4---------5---------6---------7---------8
Finora abbiamo replicato le funzionalità presenti nei \texttt{JCL} mainframe.
In queste librerie è presente anche qualche funzionalità aggiuntiva.
%--------1---------2---------3---------4---------5---------6---------7---------8

%--------1---------2---------3---------4---------5---------6---------7---------8
\begin{elisting}[!htb]
\begin{javacode}
public Integer call() {
    return JCL.getInstance().job("job01")
        .forkPgm("sort1", this::sort1)
        .forkPgm("sort2", this::sort2)
        .join()
        .cond(0,NE).execPgm("balance", this::balance)
        .complete();
}
\end{javacode}
\caption{Esempio uso di una procedura nel job}
\label{lst:demoFork}
\end{elisting}
%--------1---------2---------3---------4---------5---------6---------7---------8
È possibile lanciare un programma o una procedura in background in modo da
eseguire più step in parallelo, nel lis.~\ref{lst:demoFork} vengono lanciati due
step in background (\texttt{sort1} e \texttt{sort2}), si attende che tutti i
programmi in background siano finiti, e se non ci sono stati errori, viene
eseguito il terzo step (\texttt{balance}).
%--------1---------2---------3---------4---------5---------6---------7---------8


%--------1---------2---------3---------4---------5---------6---------7---------8
\begin{elisting}[!htb]
\begin{Verbatim}[fontsize=\small,frame=single]
   Name   !  rc  !     Date-Time Start/Skip      !         Date-Time End         !       Lapse
----------+------+-------------------------------+-------------------------------+-------------------
sort1&....!    0 ! 2024-05-17T15:00:58.951504    ! 2024-05-17T15:01:00.799122    ! 00:00:01.847618
join()....!    0 ! 2024-05-17T15:00:58.951838    ! 2024-05-17T15:01:00.810442    ! 00:00:01.858604
sort2&....!    0 ! 2024-05-17T15:00:58.952275    ! 2024-05-17T15:00:59.265962    ! 00:00:00.313687
balance...!    0 ! 2024-05-17T15:01:00.811173    ! 2024-05-17T15:01:01.065912    ! 00:00:00.254739
----------+------+-------------------------------+-------------------------------+-------------------
job01.....!    0 ! 2024-05-17T15:00:58.950293    ! 2024-05-17T15:01:01.066175    ! 00:00:02.115882
\end{Verbatim}
\caption{Esempio report fine esecuzione job con step paralleli}
\label{lst:logFork}
\end{elisting}
%--------1---------2---------3---------4---------5---------6---------7---------8
Come si vede dal log prodotto, lis.~\ref{lst:logFork}, i primi due step sono
stati eseguiti in background, \texttt{join} ha atteso che tutti i programmi in
background terminassero e ha restituito come \textit{returnCode} il massimo tra
tutti quelli restituiti dai programmi in background terminati, non essendoci
errori è stato eseguito il terzo step.
%--------1---------2---------3---------4---------5---------6---------7---------8

%--------1---------2---------3---------4---------5---------6---------7---------8
Nel mondo mainframe i parametri del job (degli step) sono scalari, nella classe
parametro possono essere presenti anche strutture più complesse, inoltre la
classe parametro non è necessariamente solo una classe che fornisce valori di
input per gli step, ma può essere usata dagli step come contesto in cui passare
valori agli step successivi.
In particolare se la classe parametro \texttt{<P extends Iterable<Q>, Q>} è
possibile iterare uno step (o una procedura) usando la classe \verb!Q! come
parametro dello step (o della procedura).
%--------1---------2---------3---------4---------5---------6---------7---------8

%--------1---------2---------3---------4---------5---------6---------7---------8
\begin{elisting}[!htb]
\begin{javacode}
public Integer call() {
    List<String> ls = new ArrayList<>();
    return JCL.getInstance().job("job01")
        .execPgm(ls, "list", step01::run)
        .cond(0,NE).forEachPgm(ls, s->s, step02::run)
        .complete();
}
\end{javacode}
\caption{Esempio uso di loop di un programma}
\label{lst:demoLoop}
\end{elisting}
%--------1---------2---------3---------4---------5---------6---------7---------8
Nel lis.~\ref{lst:demoLog} è mostrato un esempio di loop su un programma; come
classe parametro viene usata una lista vuota, il primo programma alimenta la
lista, e per ogni valore della lista viene eseguito il secondo programma.
%--------1---------2---------3---------4---------5---------6---------7---------8

%--------1---------2---------3---------4---------5---------6---------7---------8
Dopo questa veloce panoramica sulle funzioni disponibili per coordinare
l'\,esecuzione degli step, vediamo i le funzioni in dettaglio una per una.
%--------1---------2---------3---------4---------5---------6---------7---------8


\section{Classe statistica --- \texttt{StatsCount}}\label{sec:statCount}
%--------1---------2---------3---------4---------5---------6---------7---------8
La classe statistica utente deve estendere la classe statistica base
(\texttt{StatsCount}), mostrata nel lis.~\ref{lst:statCount}.
L'\,utente deve indicare l'\,etichetta da associare la programma in esecuzione
(nome dello step) nel costruttore della classe, e definire un metodo che
fornisca informazioni sulla elaborazione eseguita dal programma.

%--------1---------2---------3---------4---------5---------6---------7---------8
\begin{elisting}[!htb]
\begin{javacode}
public abstract class StatsCount {
    protected StatsCount(String name) { ... }
    protected abstract void recap(PrintWriter pw);
}
\end{javacode}
\caption{Costruttore e metodo astratto della classe statistica}
\label{lst:statCount}
\end{elisting}
%--------1---------2---------3---------4---------5---------6---------7---------8


\section{Operatori condizionali} \label{sec:opCond}
%--------1---------2---------3---------4---------5---------6---------7---------8
Ogni step termina con un returnCode, gli operatori condizionali permettono
di indicare se un programma o una procedura devono essere lanciati in relazione
al returnCode dell'\,ultimo step eseguito, o uno degli step precedenti eseguito.
%--------1---------2---------3---------4---------5---------6---------7---------8

%--------1---------2---------3---------4---------5---------6---------7---------8
\begin{elisting}[!htb]
\begin{javacode}
public interface CondAction<T> {
    JobAction<T> cond(int cc, Cond cond);
    JobAction<T> cond(int cc, Cond cond, String stepName);
    JobAction<T> when(int cc, Cond cond);
    JobAction<T> when(int cc, Cond cond, String stepName);
}
\end{javacode}
\caption{Interfaccia con gli operatori condizionali}
\label{lst:condActionDef}
\end{elisting}
%--------1---------2---------3---------4---------5---------6---------7---------8

%--------1---------2---------3---------4---------5---------6---------7---------8
I metodi condizionali accettano 2 o 3 argomenti, se sono presenti solo 2
argomenti il confronto è fatto con l'\,ultimo programma eseguito, se sono
presenti 3 argomenti, il confronto è fatto con il programma indicato nel terzo
argomento.
Il primo argomento è il valore che vogliamo confrontare,
il secondo argomento è la relazione, vedi lis.~\ref{lst:condDef}, tra il valore
fornito e il codice restituito dal programma eseguito.
%--------1---------2---------3---------4---------5---------6---------7---------8

%--------1---------2---------3---------4---------5---------6---------7---------8
\begin{elisting}[!htb]
\begin{javacode}
public enum Cond {
    EQ, NE, LT, GT, LE, GE
}
\end{javacode}
\caption{Enum con le relazioni degli operatori condizionali}
\label{lst:condDef}
\end{elisting}
%--------1---------2---------3---------4---------5---------6---------7---------8

%--------1---------2---------3---------4---------5---------6---------7---------8
Il metodo condizionale \texttt{cond} (copiato dagli script JCL mainframe)
permette di definire la condizione per \textbf{non eseguire} il programma
successivo in base al codice ritornato dal programma precedentemente eseguito.
Il metodo condizionale \texttt{when}, è l'\,opposto di \texttt{cond}, permette
di definire la condizione per \textbf{eseguire} il programma successivo in base
al codice ritornato dal programma precedentemente eseguito.
%--------1---------2---------3---------4---------5---------6---------7---------8


\section{Procedure} \label{sec:proc}
%--------1---------2---------3---------4---------5---------6---------7---------8
Una procedura permette di raggruppare un insieme di programmi e la relativa
logica condizionale di esecuzione.
Sono disponibili 2 costruttori statici di una procedura, lis.~\ref{lst:newProc}.
%--------1---------2---------3---------4---------5---------6---------7---------8

%--------1---------2---------3---------4---------5---------6---------7---------8
\begin{elisting}[!htb]
\begin{javacode}
public abstract class Proc<P> {
    public static <Q> Proc<Q> create(BiFunction<Q, JobStatus, JobStatus> route);
    public static Proc<Void> create(UnaryOperator<JobStatus> route);
}
\end{javacode}
\caption{Comandi per creare una procedura}
\label{lst:newProc}
\end{elisting}
%--------1---------2---------3---------4---------5---------6---------7---------8

%--------1---------2---------3---------4---------5---------6---------7---------8
La procedura può essere creata prima del job con il suo flusso di esecuzione, o
in linea nel flusso di esecuzione.
Nel flusso di esecuzione di una procedura sono disponibili tutti i comandi del
flusso di esecuzione del job, anche l'\,esecuzione di un'\,altra procedura.
%--------1---------2---------3---------4---------5---------6---------7---------8

%--------1---------2---------3---------4---------5---------6---------7---------8
\begin{elisting}[!htb]
\begin{javacode}
public interface ExecProc<S> {
    <P> S execProc(P p, String procName, Proc<P> proc);
    S execProc(String procName, Proc<Void> proc);
}
\end{javacode}
\caption{Comandi per lanciare una procedura}
\label{lst:runProc}
\end{elisting}
%--------1---------2---------3---------4---------5---------6---------7---------8

%--------1---------2---------3---------4---------5---------6---------7---------8
Per eseguire una procedura il comando è \texttt{execProc}, vedi
lis.~\ref{lst:runProc}
%--------1---------2---------3---------4---------5---------6---------7---------8


\section{Esecuzione in \textit{background}} \label{sec:fork}
%--------1---------2---------3---------4---------5---------6---------7---------8
Sia i programmi che le procedure possono essere lanciati in background,
vedi lis.~\ref{lst:forkPgm},\ref{lst:forkProc}, in questo modo è possibile
eseguire più step in parallelo.
I programmi lanciati in questo modo sono separati rispetto al job principale.
%--------1---------2---------3---------4---------5---------6---------7---------8

%--------1---------2---------3---------4---------5---------6---------7---------8
\begin{elisting}[!htb]
\begin{javacode}
public interface ForkPgm<S> {
    <P, C extends StatsCount> S forkPgm(P p, C c, BiFunction<P, C, Integer> pgm);
    <P> S forkPgm(P p, String stepName, ToIntFunction<P> pgm);
    <C extends StatsCount> S forkPgm(C c, ToIntFunction<C> pgm);
    S forkPgm(String stepName, IntSupplier pgm);
    <P, C extends StatsCount> S forkPgm(P p, C c, BiConsumer<P, C> pgm);
    <P> S forkPgm(P p, String stepName, Consumer<P> pgm);
    <C extends StatsCount> S forkPgm(C c, Consumer<C> pgm);
    S forkPgm(String stepName, Runnable pgm);
}
\end{javacode}
\caption{Interfaccia con i metodi di esecuzione di un programma (step) in \textit{background}}
\label{lst:forkPgm}
\end{elisting}
%--------1---------2---------3---------4---------5---------6---------7---------8

%--------1---------2---------3---------4---------5---------6---------7---------8
A un certo punto sarà necessario recuperare il risultato di questi programmi
lanciati in background nel flusso del job principale.
Questo è fatto con il comando \texttt{join}, che può essere usato un due modi,
vedi lis.~\ref{lst:joinAware}.
Senza argomenti attende che tutti i programmi che sono stati lanciati in
background siano terminati, e restituisce il \textsl{returnCode} massimo,
con l'\,argomento (il nome dello step associato al programma in esecuzione)
attende che sia terminato il programma indicato e restituisce il suo
\textsl{returnCode}.
Nel caso che il risultato di un programma che sta girando in background non
sia più necessario è possibile tentare di interrompere la sua esecuzione e
ignorare il suo \textsl{returnCode} con il comando \texttt{quit}.
%--------1---------2---------3---------4---------5---------6---------7---------8

%--------1---------2---------3---------4---------5---------6---------7---------8
\begin{elisting}[!htb]
\begin{javacode}
public interface ForkProc<S> {
    <P> S forkProc(P p, String procName, Proc<P> proc);
    S forkProc(String procName, Proc<Void> proc);
}
\end{javacode}
\caption{Interfaccia con i metodi di esecuzione di una procedura in \textit{background}}
\label{lst:forkProc}
\end{elisting}
%--------1---------2---------3---------4---------5---------6---------7---------8ù

%--------1---------2---------3---------4---------5---------6---------7---------8
\begin{elisting}[!htb]
\begin{javacode}
public interface Joinable {
    JobStatus join();
    JobStatus join(String name);
    JobStatus quit(String name);
}
\end{javacode}
\caption{Interfaccia con i metodi controllare l'\,esecuzione di un programma o procedura in \textit{background}}
\label{lst:joinAware}
\end{elisting}
%--------1---------2---------3---------4---------5---------6---------7---------8


\section{Loop} \label{sec:loop}
%--------1---------2---------3---------4---------5---------6---------7---------8


%--------1---------2---------3---------4---------5---------6---------7---------8
\begin{elisting}[!htb]
\begin{javacode}
public interface LoopPgm<S> {
    <P extends Iterable<Q>, Q, C extends StatsCount> S forEachPgm(P p, Function<Q, C> c,
                                                                  BiFunction<Q, C, Integer> pgm);
    <P extends Iterable<Q>, Q> S forEachPgm(P p, Function<Q, String> name, ToIntFunction<Q> pgm);
    <P extends Iterable<Q>, Q, C extends StatsCount> S forEachPgm(P p, Function<Q, C> c, BiConsumer<Q, C> pgm);
    <P extends Iterable<Q>, Q> S forEachPgm(P p, Function<Q, String> name, Consumer<Q> pgm);
}
\end{javacode}
\caption{Interfaccia elaborazione ripetuta di un programma}
\label{lst:loopPgm}
\end{elisting}
%--------1---------2---------3---------4---------5---------6---------7---------8

%--------1---------2---------3---------4---------5---------6---------7---------8
\begin{elisting}[!htb]
\begin{javacode}
public interface LoopProc<S> {
    <P extends Iterable<Q>, Q> S forEachProc(P p, Function<Q, String> name, Proc<Q> proc);
}
\end{javacode}
\caption{Interfaccia elaborazione ripetuta di una procedura}
\label{lst:loopProc}
\end{elisting}
%--------1---------2---------3---------4---------5---------6---------7---------8
