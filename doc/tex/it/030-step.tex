\chapter{Step}

%--------1---------2---------3---------4---------5---------6---------7---------8
In una elaborazione di uno step in generale abbiamo una (o più) origine dati
(tipicamente un file, ma non solo), una (o più) destinazione dati (tipicamente
un file, ma non solo) e un processo elaborativo che prende i dati dalla origine,
li trasforma in modo da poterli scrivere nella destinazione.

%--------1---------2---------3---------4---------5---------6---------7---------8
Nel caso che ci siano più origini dati, è il processo elaborativo che deve
decidere da quale origine leggere, e non è possibile parallelizzare il processo
elaborativo.

%--------1---------2---------3---------4---------5---------6---------7---------8
Nel caso di una singola origine dati il processo elaborativo non richiede una
logica di lettura; la lettura dalla origine dati può essere fatta dalla
infrastruttura, e il processo elaborativo può essere parallelizzato.
In questo caso, nelle destinazioni dati può essere mantenuta o meno la
sequenzialità della origine dati (dipende dalla implementazione della
parallelizzazione).

%--------1---------2---------3---------4---------5---------6---------7---------8
Per gestire in modo generico le sorgenti e destinazioni dati vengono usate le
classi
\begin{itemize}
    \item \hyperref[sec:srcRes]{\texttt{SourceResource}} sorgente dati
    \item \hyperref[sec:snkRes]{\texttt{SinkResource}} destinazione dati
\end{itemize}

%--------1---------2---------3---------4---------5---------6---------7---------8
Nel caso che sia presente una sola sorgente dati la parte elaborativa può
produrre una tupla e sarà la infrastruttura a inviare i dati alle destinazioni,
in questo caso il formato elaborativo è:
%--------1---------2---------3---------4---------5---------6---------7---------8
\begin{elisting}[!htb]
\begin{javacode}
    Pgm.from(src).into(snk1,snk2).forEach(it -> { ... });
\end{javacode}
\caption{elaborazione $1\mapsto N$}
\label{lst:processTuple}
\end{elisting}


%--------1---------2---------3---------4---------5---------6---------7---------8
in alternativa alla parte elaborativa oltre al valore di input vengono forniti i
\texttt{Consumer} per scrivere direttamente in dati sulle corrispondenti
destinazioni, il questo caso il formato elaborativo è:
%--------1---------2---------3---------4---------5---------6---------7---------8
\begin{elisting}[!htb]
\begin{javacode}
    Pgm.from(src).into(snk1,snk2).forEach((it,wr1,wr2) -> { ... });
\end{javacode}
\caption{elaborazione writer}
\label{lst:processWriter}
\end{elisting}
%--------1---------2---------3---------4---------5---------6---------7---------8
entrambi i formati possono essere parallelizzati, ma solo il primo permette una
implementazione che mantiene l'\,ordine dei dati originale.

%--------1---------2---------3---------4---------5---------6---------7---------8
Nel caso che siano presenti più di una sorgente dati, alla parte elaborativa
vengono forniti i \texttt{Supplier} per leggere i dati dalle origini, e i
\texttt{Consumer} per scrivere i dati, in questo caso il formato elaborativo è:
%--------1---------2---------3---------4---------5---------6---------7---------8
\begin{elisting}[!htb]
\begin{javacode}
    Pgm.from(src1, src2).into(snk1, snk2).proceed((rd1, rd2, wr1, wr2) -> { ... });
\end{javacode}
\caption{elaborazione reader-writer}
\label{lst:pullProcess}
\end{elisting}


%--------1---------2---------3---------4---------5---------6---------7---------8
Possono essere utilizzata da 1 a 3 sorgenti e da 0 a 8 destinazioni.

%--------1---------2---------3---------4---------5---------6---------7---------8
Nel caso di elaborazione sequenziale la separazione della elaborazione in una
parte dedicata alla lettura dei dati, una dedicata alla scrittura dei dati e una
specifica per la elaborazione dei dati, non offre particolari vantaggi.

%--------1---------2---------3---------4---------5---------6---------7---------8
Questa separazione è propedeutica alla elaborazione parallela.
Una elaborazione sequenziale nella forma~\ref{lst:pushSeq}
%--------1---------2---------3---------4---------5---------6---------7---------8
\begin{elisting}[!htb]
\begin{javacode}
    Pgm.from(src).into(...).forEach(...);
\end{javacode}
\caption{elaborazione sequenziale}
\label{lst:pushSeq}
\end{elisting}
%--------1---------2---------3---------4---------5---------6---------7---------8
può essere trasformata nella forma parallela~\ref{lst:pushMt}, sostituendo il
\texttt{forEach} in \texttt{forEachParallel}:
%--------1---------2---------3---------4---------5---------6---------7---------8
\begin{elisting}[!htb]
\begin{javacode}
    Pgm.from(src).into(...).forEachParallel(numTask, ...);
\end{javacode}
\caption{elaborazione multitask}
\label{lst:pushMt}
\end{elisting}
%--------1---------2---------3---------4---------5---------6---------7---------8
senza nessuna altra modifica.

\VerbatimFootnotes


\section{Sorgente dati --- \texttt{SourceResource}} \label{sec:srcRes}
%--------1---------2---------3---------4---------5---------6---------7---------8
Una generica sorgente dati è una classe che implementa l'\,interfaccia
\texttt{SourceResource}, vedi lis.~\ref{lst:srcRef}, dove il metodo \texttt{get}
fornisce la risorsa \texttt{U} che può essere usata in un
\texttt{try-with-resources}, e i metodi \texttt{iterator} e \texttt{supplier}
permettono di \textsl{leggere} gli elementi mediante un loop\footnote{%
    uso di una \texttt{SourceResource} con un \texttt{Iterator}:
    \begin{javacode}
        try (U u = source.get()) {
        Iterator<I> iterator = source.iterator(u);
        while (iterator.hasNext()) {
            I i = iterator.next();
            // TODO
        }
    }
    \end{javacode}
} o dei supplier\footnote{%
    uso di una \texttt{SourceResource} con un \texttt{Supplier}:
    \begin{javacode}
        try (U u = source.get()) {
        Supplier<I> supplier = source.supplier(u);
        I i = supplier.get();
        while (i != null) {
            // TODO
            i = supplier.get();
        }
    }
    \end{javacode}
}.
%--------1---------2---------3---------4---------5---------6---------7---------8

%--------1---------2---------3---------4---------5---------6---------7---------8
\begin{elisting}[!htb]
\begin{javacode}
public interface SourceResource<U extends AutoCloseable, I> {
    U get();
    Iterator<I> iterator(U u);
    Supplier<I> supplier(U u);
}
\end{javacode}
\caption{Metodi per aprire la risorsa e leggere i dati}
\label{lst:srcRef}
\end{elisting}
%--------1---------2---------3---------4---------5---------6---------7---------8

%--------1---------2---------3---------4---------5---------6---------7---------8
L'\,interfaccia oltre a indicare i metodi di istanza offre dei costruttori
statici facilitare l'\,implementazione della interfaccia.
%--------1---------2---------3---------4---------5---------6---------7---------8

%--------1---------2---------3---------4---------5---------6---------7---------8
\begin{elisting}[!htb]
\begin{javacode}
public interface SourceResource<U extends AutoCloseable, I> {
    static SourceResource<BufferedReader, String> bufferedReader(File file, Charset cs, Runnable inc);
    static SourceResource<BufferedReader, String> bufferedReader(File file, Runnable inc);
    static SourceResource<BufferedReader, String> bufferedReader(File file, Charset cs);
    static SourceResource<BufferedReader, String> bufferedReader(File file);
}
\end{javacode}
\caption{Metodi creare una risorsa \texttt{SourceResource<BufferedReader, String>}}
\label{lst:srcRefBr}
\end{elisting}
%--------1---------2---------3---------4---------5---------6---------7---------8
%--------1---------2---------3---------4---------5---------6---------7---------8
Se leggiamo un file, una riga per volta, possiamo usare il metodo statico
\texttt{bufferedReader}, sono disponibili 4 varianti, lis.~\ref{lst:srcRefBr}.
L'\,argomento \texttt{file} è obbligatorio, è possibile indicare un charset per
la lettura del file, ed è possibile indicare un'\,azione da eseguire dopo aver
letto una riga dal file (normalmente un contatore per le statistiche di
esecuzione).
%--------1---------2---------3---------4---------5---------6---------7---------8

%--------1---------2---------3---------4---------5---------6---------7---------8
\begin{elisting}[!htb]
\begin{javacode}
public interface SourceResource<U extends AutoCloseable, I> {
    static <I> SourceResource<BufferedReader, I> bufferedReader(File file, Function<String,I> dec,
                                                                Charset cs, Runnable inc);
    static <I> SourceResource<BufferedReader, I> bufferedReader(File file, Function<String,I> dec, Charset cs);
    static <I> SourceResource<BufferedReader, I> bufferedReader(File file, Function<String,I> dec, Runnable inc);
    static <I> SourceResource<BufferedReader, I> bufferedReader(File file, Function<String,I> dec);
}
\end{javacode}
\caption[Metodi creare una risorsa \texttt{SourceResource<BufferedReader,I>}
con funzione di decodifica]{Metodi creare una risorsa \texttt{SourceResource<BufferedReader,I>}
con funzione di decodifica \texttt{String}~$\mapsto$~\texttt{I}}
\label{lst:srcRefBrDec}
\end{elisting}
%--------1---------2---------3---------4---------5---------6---------7---------8
Se il file viene letto una riga per volta, e la riga viene deserializzata nella
classe dati, può essere usato un costruttore nella forma
lis.~\ref{lst:srcRefBrDec}, che sono analoghi alla forma precedente, ma hanno
la funzione di decodifica in modo che l'\,iterator o il supplier forniscano
direttamente la classe dati.
%--------1---------2---------3---------4---------5---------6---------7---------8

%--------1---------2---------3---------4---------5---------6---------7---------8
\begin{elisting}[!htb]
\begin{javacode}
public interface SourceResource<U extends AutoCloseable, I> {
    static <U extends AutoCloseable, I> SourceResource<U, I> fromIterator(Supplier<U> ctor,
                                                                          Function<U, Iterator<I>> reader);
    static <U extends AutoCloseable & Iterable<I>, I> SourceResource<U, I> fromIterator(Supplier<U> ctor);
    static <I> SourceResource<?, I> fromIterator(Iterable<I> iterable);
}
\end{javacode}
\caption{Metodi per creare una origine dati da un iterator}
\label{lst:srcRefIter}
\end{elisting}
%--------1---------2---------3---------4---------5---------6---------7---------8
Infine esistono i costruttori statici generici mediante \texttt{Iterator},
lis.~\ref{lst:srcRefIter}, e \texttt{Supplier}, lis.~\ref{lst:srcRefSupp}.
%--------1---------2---------3---------4---------5---------6---------7---------8

%--------1---------2---------3---------4---------5---------6---------7---------8
\begin{elisting}[!htb]
\begin{javacode}
public interface SourceResource<U extends AutoCloseable, I> {
    static <U extends AutoCloseable, I> SourceResource<U, I> fromSupplier(
        Supplier<U> ctor, Function<U, Supplier<I>> reader);
    static <U extends AutoCloseable & Supplier<I>, I> SourceResource<U, I> fromSupplier(Supplier<U> ctor);
}
\end{javacode}
\caption{Metodi per creare una origine dati da un supplier}
\label{lst:srcRefSupp}
\end{elisting}
%--------1---------2---------3---------4---------5---------6---------7---------8


\section{Destinazione dati --- \texttt{SinkResource}} \label{sec:snkRes}
%--------1---------2---------3---------4---------5---------6---------7---------8
Una generica destinazione dati è una classe che implementa l'\,interfaccia
\texttt{SinkResource}, vedi lis.~\ref{lst:snkRef}, dove il metodo \texttt{get}
fornisce la risorsa \texttt{U} che può essere usata con un
\texttt{try-with-resources} ed il metodo \texttt{accept} scrive\footnote{%
    uso di una \texttt{SinkResource}:
    \begin{javacode}
        try (T t = sink.get()) {    // open resource
        O o = // TODO
        sink.accept(t, o);      // write data
    }
    \end{javacode}
} il dato.
%--------1---------2---------3---------4---------5---------6---------7---------8
L'\,interfaccia oltre a indicare i metodi di istanza offre dei costruttori
statici per facilitare l'\,implementazione dell'\,interfaccia.
%--------1---------2---------3---------4---------5---------6---------7---------8

%--------1---------2---------3---------4---------5---------6---------7---------8
\begin{elisting}[!htb]
\begin{javacode}
public interface SinkResource<U extends AutoCloseable, O> {
    U get();
    void accept(U u, O o);
}
\end{javacode}
\caption{Metodi per aprire la risorsa e scrivere i dati}
\label{lst:snkRef}
\end{elisting}
%--------1---------2---------3---------4---------5---------6---------7---------8

%--------1---------2---------3---------4---------5---------6---------7---------8
Se la risorsa è un file e viene scritta una riga per volta possono essere usato
il metodo statico \texttt{bufferedWriter}, sono disponibili 4 varianti,
lis.~\ref{lst:snkRefBw}. L'\,argomento \texttt{file} è obbligatorio, è possibile
indicare il charset per la scrittura del file, e una azione da eseguire dopo
aver scritto la riga del file (normalmente un contatore per le statistiche di
esecuzione).
%--------1---------2---------3---------4---------5---------6---------7---------8

%--------1---------2---------3---------4---------5---------6---------7---------8
\begin{elisting}[!htb]
\begin{javacode}
public interface SinkResource<U extends AutoCloseable, O> {
    static SinkResource<BufferedWriter, String> bufferedWriter(File file, Charset cs, Runnable inc);
    static SinkResource<BufferedWriter, String> bufferedWriter(File file, Runnable inc);
    static SinkResource<BufferedWriter, String> bufferedWriter(File file, Charset cs);
    static SinkResource<BufferedWriter, String> bufferedWriter(File file);
}
\end{javacode}
\caption{Metodi per creare una risorsa \texttt{SinkResource<BufferedWriter, String>}}
\label{lst:snkRefBw}
\end{elisting}
%--------1---------2---------3---------4---------5---------6---------7---------8

%--------1---------2---------3---------4---------5---------6---------7---------8
Se la riga sul file rappresenta la serializzazione di un oggetto, possono essere
usati i costruttori statici nella forma lis.~\ref{lst:snkRefBwEnc}, che sono
analoghi alla forma precedente, ma hanno la funzione di codifica della classe
in stringa, in modo che il \textit{writer} possa scrivere direttamente la classe
dati.
%--------1---------2---------3---------4---------5---------6---------7---------8

%--------1---------2---------3---------4---------5---------6---------7---------8
\begin{elisting}[!htb]
\begin{javacode}
public interface SinkResource<U extends AutoCloseable, O> {
    static <O> SinkResource<BufferedWriter, O> bufferedWriter(File file, Function<O,String> enc, Charset cs,
                                                              Runnable inc);
    static <O> SinkResource<BufferedWriter, O> bufferedWriter(File file, Function<O,String> enc, Charset cs);
    static <O> SinkResource<BufferedWriter, O> bufferedWriter(File file, Function<O,String> enc, Runnable inc);
    static <O> SinkResource<BufferedWriter, O> bufferedWriter(File file, Function<O,String> enc);
}
\end{javacode}
\caption[Metodi per creare una risorsa \texttt{SinkResource<BufferedWriter,O>}
con funzione di codifica]{Metodi per creare una risorsa \texttt{SinkResource<BufferedWriter,O>}
con funzione di codifica \textit{O}~$\mapsto$~\texttt{String}}
\label{lst:snkRefBwEnc}
\end{elisting}
%--------1---------2---------3---------4---------5---------6---------7---------8

%--------1---------2---------3---------4---------5---------6---------7---------8
Infine esistono i costruttori statici generici, lis.~\ref{lst:snkRefGen}.
Il parametro \texttt{action} indica una azione che viene eseguita dopo la
scrittura del dato (normalmente un contatore per le statistiche di esecuzione).
%--------1---------2---------3---------4---------5---------6---------7---------8

%--------1---------2---------3---------4---------5---------6---------7---------8
\begin{elisting}[!htb]
\begin{javacode}
public interface SinkResource<U extends AutoCloseable, O> {
    static <U extends AutoCloseable, O> SinkResource<U, O> of(Supplier<U> ctor, BiConsumer<U, O> writer);
    static <U extends AutoCloseable, O> SinkResource<U, O> of(Supplier<U> ctor, BiConsumer<U, O> writer,
                                                              Runnable action);
    static <O> SinkResource<?, O> of(Consumer<O> writer);
    static <O> SinkResource<?, O> of(Consumer<O> writer, Runnable action);
}
\end{javacode}
\caption{Metodi per creare una risorsa \texttt{SinkResource} generica}
\label{lst:snkRefGen}
\end{elisting}
%--------1---------2---------3---------4---------5---------6---------7---------8
