\section*{Introduzione}
%--------1---------2---------3---------4---------5---------6---------7---------8
Nel mondo mainframe una elaborazione batch normalmente è divisa in più parti
(step), e questi sono coordinati da uno script di controllo (JCL).
%--------1---------2---------3---------4---------5---------6---------7---------8
Questa libreria mette a disposizione delle classi per facilitare lo sviluppo
delle elaborazioni dei singoli step e di coordinale l'\,esecuzione degli step.
%--------1---------2---------3---------4---------5---------6---------7---------8

%--------1---------2---------3---------4---------5---------6---------7---------8
Un esempio minimale di batch con un singolo step è:
%--------1---------2---------3---------4---------5---------6---------7---------8
\begin{elisting}[!htb]
\begin{javacode}
public Integer call() {
    return JCL.getInstance().job("job01")
        .execPgm("step01", step01::run)
        .complete();
}
\end{javacode}
\caption{esempio minimale job}
\label{lst:demoJob}
\end{elisting}
%--------1---------2---------3---------4---------5---------6---------7---------8
dove \texttt{job01} è il nome del job, \texttt{step01} è il come dello step e
\texttt{step01::run} è il riferimento al metodo che esegue lo step.
%--------1---------2---------3---------4---------5---------6---------7---------8

%--------1---------2---------3---------4---------5---------6---------7---------8
L'\,esecuzione del job produce nel log un report finale del tipo:
%--------1---------2---------3---------4---------5---------6---------7---------8
\begin{elisting}[!htb]
\begin{Verbatim}[fontsize=\small,frame=single]
   Name   !  rc  !     Date-Time Start/Skip      !         Date-Time End         !       Lapse
---------+------+-------------------------------+-------------------------------+-------------------
step01...!    0 ! 2024-05-17T14:59:12.03243396  ! 2024-05-17T14:59:12.590157083 ! 00:00:00.557723123
---------+------+-------------------------------+-------------------------------+-------------------
job01....!    0 ! 2024-05-17T14:59:12.028019213 ! 2024-05-17T14:59:12.590998545 ! 00:00:00.562979332
\end{Verbatim}
\caption{esempio report fine esecuzione job}
\label{lst:demoLog}
\end{elisting}
